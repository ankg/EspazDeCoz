\documentclass[10pt]{article}
\usepackage{calc}
\usepackage[bottom,stable]{footmisc}
\usepackage{comment}
\makeatletter
\newlength{\bibhang}
\setlength{\bibhang}{1em} %1em}
\newlength{\bibsep}
 {\@listi \global\bibsep\itemsep \global\advance\bibsep by\parsep}
\newenvironment{bibsection}%
        {\begin{enumerate}{}{%
%        {\begin{list}{}{%
       \setlength{\leftmargin}{\bibhang}%
       \setlength{\itemindent}{-\leftmargin}%
       \setlength{\itemsep}{\bibsep}%
       \setlength{\parsep}{\z@}%
        \setlength{\partopsep}{0pt}%
        \setlength{\topsep}{0pt}}}
        {\end{enumerate}\vspace{-.6\baselineskip}}
%        {\end{list}\vspace{-.6\baselineskip}}
\makeatother

\reversemarginpar

\usepackage[paper=letterpaper,
            %includefoot, % Uncomment to put page number above margin
            marginparwidth=1.2in,     % Length of section titles
            marginparsep=.05in,       % Space between titles and text
            margin=1in,               % 1 inch margins
            includemp]{geometry}

\setlength{\parindent}{0in}

\usepackage[shortlabels]{enumitem}

\usepackage{fancyhdr,lastpage}
\pagestyle{fancy}
\pagestyle{empty}      % Uncomment this to get rid of page numbers
\fancyhf{}\renewcommand{\headrulewidth}{0pt}
\fancyfootoffset{\marginparsep+\marginparwidth}
\newlength{\footpageshift}
\setlength{\footpageshift}
          {0.5\textwidth+0.5\marginparsep+0.5\marginparwidth-2in}
\lfoot{\hspace{\footpageshift}%
       \parbox{4in}{\, \hfill %
                    \arabic{page} of \protect\pageref*{LastPage} % +LP
%                    \arabic{page}                               % -LP
                    \hfill \,}}

\usepackage{xcolor,hyperref}
\definecolor{darkblue}{rgb}{0.0,0.0,0.3}
\hypersetup{colorlinks,breaklinks,
            linkcolor=darkblue,urlcolor=purple,
            anchorcolor=darkblue,citecolor=darkblue}

\newcommand{\makeheading}[2][]%
        {\hspace*{-\marginparsep minus \marginparwidth}%
         \begin{minipage}[t]{\textwidth+\marginparwidth+\marginparsep}%
             {\large \bfseries #2 \hfill #1}\\[-0.15\baselineskip]%
                 \rule{\columnwidth}{1pt}%
         \end{minipage}}

\renewcommand{\section}[1]{\pagebreak[3]%
    \hyphenpenalty=10000%
    \vspace{1.3\baselineskip}%
    \phantomsection\addcontentsline{toc}{section}{#1}%
    \noindent\llap{\scshape\smash{\parbox[t]{\marginparwidth}{\raggedright #1}}}%
    \vspace{-\baselineskip}\par}

\newenvironment{outerlist}[1][\enskip\textbullet]%
        {\begin{itemize}[#1,leftmargin=*]}{\end{itemize}%
         \vspace{-.6\baselineskip}}

\newenvironment{lonelist}[1][\enskip\textbullet]%
        {\begin{list}{#1}{%
        \setlength{\partopsep}{0pt}%
        \setlength{\topsep}{0pt}}}
        {\end{list}\vspace{-.6\baselineskip}}

\newenvironment{innerlist}[1][\enskip\textbullet]%
        {\begin{itemize}[#1,leftmargin=*,parsep=0pt,itemsep=0pt,topsep=0pt,partopsep=0pt]}
        {\end{itemize}}

\newenvironment{loneinnerlist}[1][\enskip\textbullet]%
        {\begin{itemize}[#1,leftmargin=*,parsep=0pt,itemsep=0pt,topsep=0pt,partopsep=0pt]}
        {\end{itemize}\vspace{-.6\baselineskip}}
\newcommand{\blankline}{\quad\pagebreak[3]}
\newcommand{\halfblankline}{\quad\vspace{-0.5\baselineskip}\pagebreak[3]}

\newcommand\doilink[1]{\href{http://dx.doi.org/#1}{#1}}
\newcommand\doi[1]{doi:\doilink{#1}}

\providecommand*\url[1]{\href{#1}{#1}}
\renewcommand*\url[1]{\href{#1}{\texttt{#1}}}

\providecommand*\email[1]{\href{mailto:#1}{#1}}
\providecommand\BibTeX{{B\kern-.05em{\sc i\kern-.025em b}\kern-.08em
    \TeX}}
\providecommand\Matlab{\textsc{Matlab}}


\begin{document}
\makeheading{\LARGE{Aniket Gupta}\\\normalsize Sophomore in the Department of Computer Science and Engineering,\\ 
Indian Institute of Technology, Roorkee}

\section{Contact Information}

% NOTE: Mind where the & separators and \\ breaks are in the following
%       table.
%
% ALSO: \rcollength is the width of the right column of the table
%       (adjust it to your liking; default is 1.85in).
%
\newlength{\rcollength}\setlength{\rcollength}{1.4in}%
%
\begin{tabular}[t]{@{}p{\textwidth-\rcollength}p{\rcollength}}
A-542, Rajiv Bhawan, IIT Roorkee,   & (+91)9557040676 \\
Roorkee, Uttarakhand 247667.     & \email{ankgruec@iitr.ac.in}\\
\end{tabular}

\section{Areas Of\\Interest}

Computer Vision, Artificial Intelligence, Machine Learning, Web Development, Data Structures and  Algorithms.

\section{Education}

%\href{http://www.iitk.ac.in}{{Indian Institute of Technology}},
%Roorkee, Uttarakhand.
%\begin{outerlist}

%\item[] B.Tech., {Computer Science}, \\Cumulative Grade Point Average(First Two Semesters): 7.9/10\\
%            \emph{Expected:} Summer 2017

%\end{outerlist}

%\vspace{0.1in}

D.D.E.C. Ramaipur, Kanpur, UP.

 \begin{outerlist}
\item[] All India Senior School Certificate Examination(A.I.S.S.C.E.) conducted by Central Board of Secondary Education(C.B.S.E.), 2013, Percentage obtained: 92.0
\end{outerlist}
\vspace{0.1in}

 Dr. V.S.E.C., Kanpur, UP.
\begin{outerlist}
\item[] Indian Certificate of Secondary Education(I.C.S.E.) conducted by Council for the Indian School Certificate Examinations(C.I.S.C.E.), 2011, Percentage obtained: 97.2
\end{outerlist}

\section{Academic Achievements}
%\vspace{-0.17in}
\begin{innerlist}
\item Secured an All India Rank of 641 out of 14 lakh candidates in J.E.E.-Advanced, 2013.
\item Ranked in top 0.01\% candidates in J.E.E.-Mains, 2013.
\item Awarded the Kishore Vaigyanik Protsahan Yojana(K.V.P.Y.) fellowship by the Department of Science and Technology, Government of India, for the year 2011.
\item Qualified the National Standard Examination in Physics(N.S.E.P.), 2012, placed amongst the top 0.01\% of the candidates.
\item Qualified the National Standard Examination in Astronomy(N.S.E.A.), 2012.
 Placed in top 1\% candidates of the state in the same in 2011.
\item Qualified the Regional Mathematics Olympiad(R.M.O.), 2012.
\item Secured the first position in inter-school level Mathematical Smartness Contest(MaSCon, 2011) for high school students.
\end{innerlist}

\section{Projects}

\vspace{-0.11in}
\begin{outerlist}
\item[] \textbf{Echo}, \textcolor{darkblue}{May, 2014-July, 2014}, \hfill \textit{SDSLabs}
\begin{innerlist}
\item An e-book search application which facilitates e-book search on the IIT Roorkee intranet.
\item Uses \emph{Solr} for book indexing, runs a \emph{node.js} script daily for indexing.
\item \emph{Limonade} as a framework, provides effective routing, MVC and REST implementation.
\end{innerlist}

\item[] \textbf{Barrel Shifter}, \textcolor{darkblue}{February, 2014}, \hfill \textit{Guide: Prof. Sanjeev Manhas}
\begin{innerlist}
\item Used to shift a 16-bit binary number by the specified number of bits to the left or right.
\item Coded in the language VHDL and used Xilinx  as the IDE.
\end{innerlist}

\item[] \textbf{CourseSpace}, \textcolor{darkblue}{August, 2014-}, \hfill \textit{Guide: Prof. Balasubramaniam Raman}
\begin{innerlist}
\item Ongoing project on Object Oriented Analysis and Design.
\item Is a web application that lets students interact with professors of the
institute relating the courses that interest them.
\item Uses the router \emph{Toro}.
\end{innerlist}

\end{outerlist}

\section{Relevant \\Courses}
\textbf{Computer Science}\\
CS101: Introduction to Computer Science and Engineering: Binary Logic, K-maps.\\
CS103: Introduction to Object Oriented Programming: Java and Applets.\\
CS102: Data Structures: Trees(AVL, RB), Heaps, Basic Algorithms.\\
CS106: Discrete Structures: Set theory, Groups, Rings, Fields.\\
CS291: Analysis and design of Object Oriented systems: UML, Design Methodologies.\\ CS221: Computer Architecture and Microprocessors: Single and Multicycle datapaths, CU, ALU, Processor design(x85 and x86).\\

\textbf{Mathematics}\\
MA010: Optimization Techniques: Optimizations of real world models.\\
MA001: Calculus and Linear Algebra: Surface integrals, Volume Integrals, Matrix Operations. 

\section{Technical \\Skills} 

\textbf{Programming:} C, C++, Java, Assembly Language(MIPS and x86).\\
\textbf{Web:} HTML, CSS, JavaScript, PHP, MySQL.\\
\textbf{Frameworks:} Limonade, Toro, Jquery, Bootstrap.\\
\textbf{Other Tools:} \LaTeX, Photoshop.\\
\textbf{Platforms:} Linux, Windows 8/7/XP.

\section{Positions of\\Responsibility}

\textbf{Editor, Designer and Web Co-ordinator} at \textbf{WatchOut! News Agency}, the official news magazine of I.I.T. Roorkee.\\
\textbf{Web Developer at SDSLabs}, a popular development and coding group on the forefront of innovation at I.I.T. Roorkee.

\section{Extra-curricular\\Activities}
\begin{innerlist}
\item An active spokesperson at IIT Roorkee as a part of SDSLabs: addressed the masses in numerous talks on HTML and web development, with hundreds of people attending.
\item Member of the Organising Committee of Thomso 2013, the annnual cultural festival of IIT Roorkee.
\item Attended the BRICS workshop, 2010: Made a remote-controlled car using four way switches and multi-directional motors from scratch.
\item Made various bots as a member of the School Robotics Club:\\
Line Follower, 2009: Moves on a defined path in the form of a line.\\
Ball Grabber,  2009: Searches for a ball in the surroundings and fetches it.\\
Humanoid,	   2009: A human-like robot, that talks and responds to some fixed stimuli. 
\end{innerlist}

\section{Talks/Seminars \\Attended}
\textbf{Vijyoshi, 2012:} At IISc Bangalore, for high school students in which professors from universities like Cambridge, UC Berkeley and IMSc lectured on various topics in science relating to graph theory, cryptography, biology etc.\\
\textbf{Summer Camp, 2012}: At IISER Bhopal, involving talks on many spheres of science like biology, physics and mathematics by eminent professors.\\
\textbf{Cloud Computing, 2013}: by Microsoft at IIT Roorkee, involving the evolution and necessity of cloud storage.

\end{document}

